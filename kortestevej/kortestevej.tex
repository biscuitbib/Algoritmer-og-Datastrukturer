\section{Korteste Vej}
\hrulefill
\begin{itemize}
\item Korteste vej problemet
  \begin{itemize}
  \item Optimal delstruktur
  \item Cykler i grafen
  \item Egenskaber af en korteste vej
  \end{itemize}
\item Bellman-Ford
\item Dijkstra
\end{itemize}

\newpage
\subsection{Korteste vej problemet}
Givet en \textbf{orienteret graf} $G = (V,E)$ med en vægtsfunktion$w(u,v)$ for $(u,v) \in G.E$, så går korteste vej problemet ud på at finde en vej $p$, som er en følge af kanter i $G.E$, med en minimal vægt $w(p)$, hvore vægtsfunktionen på en vej, er summen af vægtene i dens kanter.\\

Vi definerer vægten af den korteste vej fra $u$ til $v$ til at være

$$
\delta(uv) =
\begin{cases}
  min{w(p)} & \text{ hvis der findes en vej } p \text{ fra } u \text{ til } v\\
  \infty & \text{ ellers}
\end{cases}
$$

Der er 4 varianter af korteste vej problemet:
\begin{itemize}
\item \textbf{Single-pair} bestemmer den korteste afstand fra en knude $u$ til en knude $v$.
\item \textbf{Single-source} bestemmer den korteste afstand fra en knude $s$ til alle andre knuder i $G.E$.
\item \textbf{Single-target} bestemmer den korteste afstand fra alle knuder i $G.E$ til en knude $t$.
\item \textbf{All-pairs} bestemmer den korteste afstand fra alle knuder i $G.E$ til alle andre knuder i $G.E$.
\end{itemize}

De to algortimer vi kommer til at kigge på er single-source korteste vej.\\

\subsubsection{Optimal delstruktur}
Problemet at finde den korteste vej udviser optimal delstruktur.

\begin{lemma}
  Givet en orierenteret graf $G = (V,E)$ med vægtsfunktion $w(u,v)$. Lad $p = \langle v_0, v_1, ..., v_k\rangle$ være den korteste vej $v_0 \leadsto v_k$. For alle $i,j$ således at $0 \leq i \leq j \leq k$, lad $p_{ij} = \langle v_i, v_{i+1}, ..., v_j\rangle$ være en delvej $v_i \leadsto v_j$, så er $p_{ij}$ den korteste vej fra $v_i$ til $v_j$.
\end{lemma}
Altså, alle delveje af en korteste vej, er også selv korteste veje.\\

\begin{proof}
  Hvis vi kigger på delvejene $v_0 \leadsto v_i \leadsto v_j \leadsto v_k$, så har vi at
  $$w(p) = w(p_{0i}) + w(p_{ij} + w(p_{jk})$$
  Antag så, at der findes en anden vej $p'_{ij}$ således at $w(p'_{ij}) < w(p_{ij})$, så kan vi indsætte $p'_{ij}$ i $p$, og få
  $$w(p') = w(p_{oi}) + w(p'_{ij}) + w(p_{jk}) < w(p)$$
  Hvilket er i modstrid med at $p$ var en korteste vej.
\end{proof}


\subsubsection{Cykler i grafen}
Hvis vi tillader at kanter kan have en negativ vægt, så kan vi løbe ind i problemer med negative cykler i grafen.\\
Hvis grafen $G$ indeholder en cykel, som kan nås fra vores startknude $s$, hvor summen af vægtene i cyklen er negativ, så er korteste vej problemet udefinerbart. Siden cyklen har negativ sum, kan vi bare kører flere omgange om cyklen, og få en vej, som har $\delta(s,t) = -\infty$.\\

Det giver heller ikke mening at inkluderer positive cykler i den korteste vej. Hvis en vej $p = s \leadsto t$ indeholder en positiv cykel, så må der være en knude $u$ i cyklen, hvor $s \leadsto u$ og $u \leadsto t$. Da cyklen starter og slutter i $u$, kan vi lave en vej $p'$ hvor vi har fjernet alle knuderne i cyklen, undtagen $u$ fra $p$, og have $w(p') < w(p)$.\\

En korteste vej indeholder derfor ingen cykler.

\subsubsection{Egenskaber af en korteste vej}
Korteste vej algoritmerne bruger en teknik \textbf{relaxation}. Hver knude $v$ har et felt $v.d$ som er en øvre grænse for den korteste vej  $s \leadsto v$. Derudover har de også et felt $v.\pi$, som er pegeren til den foregående knude i den korteste vej, således at vi kan konstruere selve den korteste vej.\\

Alle knuder bliver initialiseret til $v.d = \infty$ og $v.\pi=\texttt{NIL}$, udover $s.d=0$. Selve relax funktionene tager to knuder $u$ og $v$, og ser om $v.d > u.d + w(u,v)$, hvis den er, bliver $v.d = u.d + w(u,v)$ og $v.\pi = u$. Altså, hvis $p_{su} + w(u,v) < p_{sv}$, så kan vi sige, at den korteste vej fra $s$ til $v$ går igennem $u$ -- $s \leadsto u \leadsto v$.\\

Vi bruger følgende egenskaber af en korteste vej:\\

\textbf{Trekants uligheden}
\begin{lemma}
  Lad $G = (V,E))$ være en orienteret graf med vægtsfunktion $w(u,v)$ på en kant $(u,v) \in G.E$. Lad $s$ være startknuden. Der gælder for alle kanter $(u,v)$
  $$\delta(s,v) \leq \delta(s,u) + w(u,v)$$.
\end{lemma}

\begin{proof}
  Antag at $p$ er den korteste vej fra $s$ til $v$, så kan der ikke findes en vej med lavere vægt fra $s$ til $v$, så derfor, hvis kanten $(u,v)$ er i $p$, så må $p$ gå igennem $u$, altså
  $p_{sv} = p_{su} + (u,v)$
\end{proof}


\textbf{En øvre grænse egenskaben}
\begin{lemma}
  Lad $G = (V,E))$ være en orienteret graf med vægtsfunktion $w(u,v)$ på en kant $(u,v) \in G.E$. Lad $s$ være startknuden. Lad $s$ være startknuden. grafen bliver intitialiseret således at alle knuder $v \neq s \in G.V$ har $v.d = \infty$. Der gælder at $v.d \geq \delta(s,v)$, som invariant over alle relaxation operationer. Når $v.d$ får sin nedre grænse $\delta(s,v)$ forbliver den uændret. 
\end{lemma}

\begin{proof}
\end{proof}


\textbf{Ingen vej egenskaben}
\begin{corollary}
\end{corollary}

\begin{proof}
\end{proof}


\textbf{Konvergens egenskaben}
\begin{lemma}
\end{lemma}

\begin{proof}
\end{proof}


\textbf{Vej-relaxation egenskaben}
\begin{lemma}
\end{lemma}

\begin{proof}
\end{proof}


\textbf{Forgænger delgraf egenskaben}
\begin{lemma}
\end{lemma}

\begin{proof}
\end{proof}

%%% Local Variables:
%%% mode: latex
%%% TeX-master: "master"
%%% End:
