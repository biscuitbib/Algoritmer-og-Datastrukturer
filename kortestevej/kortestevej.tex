\section{Korteste Vej}
\hrulefill
\begin{itemize}
\item Korteste vej problemet
  \begin{itemize}
  \item Optimal delstruktur
  \item Cykler i grafen
  \item Egenskaber af en korteste vej
  \end{itemize}
\item Bellman-Ford
  \begin{itemize}
  \item Køretid
  \item Korrekthed
  \end{itemize}
\item Dijkstra
  \begin{itemize}
  \item Korrekthed
  \item Køretid
  \end{itemize}
\end{itemize}

\newpage
\subsection{Korteste vej problemet}
Givet en \textbf{orienteret graf} $G = (V,E)$ med en vægtsfunktion$w(u,v)$ for $(u,v) \in G.E$, altså $w : E \rightarrow \mathbb{R}$, så går korteste vej problemet ud på at finde en vej $p$, som er en følge af kanter i $G.E$, med en minimal vægt $w(p)$, hvore vægtsfunktionen på en vej, er summen af vægtene i dens kanter.\\

Vi definerer vægten af den korteste vej fra $u$ til $v$ til at være

$$
\delta(uv) =
\begin{cases}
  min{w(p)} & \text{ hvis der findes en vej } p \text{ fra } u \text{ til } v\\
  \infty & \text{ ellers}
\end{cases}
$$

Der er 4 varianter af korteste vej problemet:
\begin{itemize}
\item \textbf{Single-pair} bestemmer den korteste afstand fra en knude $u$ til en knude $v$.
\item \textbf{Single-source} bestemmer den korteste afstand fra en knude $s$ til alle andre knuder i $G.E$.
\item \textbf{Single-target} bestemmer den korteste afstand fra alle knuder i $G.E$ til en knude $t$.
\item \textbf{All-pairs} bestemmer den korteste afstand fra alle knuder i $G.E$ til alle andre knuder i $G.E$.
\end{itemize}

De to algortimer vi kommer til at kigge på er single-source korteste vej.\\

\subsubsection{Optimal delstruktur}
Problemet at finde den korteste vej udviser optimal delstruktur.

\begin{lemma}
  Givet en orierenteret graf $G = (V,E)$ med vægtsfunktion $w(u,v)$. Lad $p = \langle v_0, v_1, ..., v_k\rangle$ være den korteste vej $v_0 \leadsto v_k$. For alle $i,j$ således at $0 \leq i \leq j \leq k$, lad $p_{ij} = \langle v_i, v_{i+1}, ..., v_j\rangle$ være en delvej $v_i \leadsto v_j$, så er $p_{ij}$ den korteste vej fra $v_i$ til $v_j$.
\end{lemma}
Altså, alle delveje af en korteste vej, er også selv korteste veje.\\

\begin{proof}
  Hvis vi kigger på delvejene $v_0 \leadsto v_i \leadsto v_j \leadsto v_k$, så har vi at
  $$w(p) = w(p_{0i}) + w(p_{ij} + w(p_{jk})$$
  Antag så, at der findes en anden vej $p'_{ij}$ således at $w(p'_{ij}) < w(p_{ij})$, så kan vi indsætte $p'_{ij}$ i $p$, og få
  $$w(p') = w(p_{oi}) + w(p'_{ij}) + w(p_{jk}) < w(p)$$
  Hvilket er i modstrid med at $p$ var en korteste vej.
\end{proof}


\subsubsection{Cykler i grafen}
Hvis vi tillader at kanter kan have en negativ vægt, så kan vi løbe ind i problemer med negative cykler i grafen.\\
Hvis grafen $G$ indeholder en cykel, som kan nås fra vores startknude $s$, hvor summen af vægtene i cyklen er negativ, så er korteste vej problemet udefinerbart. Siden cyklen har negativ sum, kan vi bare kører flere omgange om cyklen, og få en vej, som har $\delta(s,t) = -\infty$.\\

Det giver heller ikke mening at inkluderer positive cykler i den korteste vej. Hvis en vej $p = s \leadsto t$ indeholder en positiv cykel, så må der være en knude $u$ i cyklen, hvor $s \leadsto u$ og $u \leadsto t$. Da cyklen starter og slutter i $u$, kan vi lave en vej $p'$ hvor vi har fjernet alle knuderne i cyklen, undtagen $u$ fra $p$, og have $w(p') < w(p)$.\\

En korteste vej indeholder derfor ingen cykler.

\subsubsection{Egenskaber af en korteste vej}
Korteste vej algoritmerne bruger en teknik \textbf{relaxation}. Hver knude $v$ har et felt $v.d$ som er en øvre grænse for den korteste vej  $s \leadsto v$. Derudover har de også et felt $v.\pi$, som er pegeren til den foregående knude i den korteste vej, således at vi kan konstruere selve den korteste vej.\\

Alle knuder bliver initialiseret til $v.d = \infty$ og $v.\pi=\texttt{NIL}$, udover $s.d=0$. Selve relax funktionene tager to knuder $u$ og $v$, og ser om $v.d > u.d + w(u,v)$, hvis den er, bliver $v.d = u.d + w(u,v)$ og $v.\pi = u$. Altså, hvis $p_{su} + w(u,v) < p_{sv}$, så kan vi sige, at den korteste vej fra $s$ til $v$ går igennem $u$ -- $s \leadsto u \leadsto v$.\\

Vi bruger følgende egenskaber af en korteste vej:\\

\begin{itemize}
\item \textbf{Trekants uligheden}
  \begin{lemma}
    Lad $G = (V,E))$ være en orienteret graf med vægtsfunktion $w(u,v)$ på en kant $(u,v) \in G.E$. Lad $s$ være startknuden. Der gælder for alle kanter $(u,v)$
    $$\delta(s,v) \leq \delta(s,u) + w(u,v)$$.
  \end{lemma}

  \begin{proof}
    Antag at $p$ er den korteste vej fra $s$ til $v$, så kan der ikke findes en vej med lavere vægt fra $s$ til $v$, så derfor, hvis kanten $(u,v)$ er i $p$, så må $p$ gå igennem $u$, altså
    $p_{sv} = p_{su} + (u,v)$
  \end{proof}


\item \textbf{En øvre grænse egenskaben}
  \begin{lemma}
    Lad $G = (V,E))$ være en orienteret graf med vægtsfunktion $w(u,v)$ på en kant $(u,v) \in G.E$. Lad $s$ være startknuden. Lad $s$ være startknuden. grafen bliver intitialiseret således at alle knuder $v \neq s \in G.V$ har $v.d = \infty$. Der gælder at $v.d \geq \delta(s,v)$, som invariant over alle relaxation operationer. Når $v.d$ får sin nedre grænse $\delta(s,v)$ forbliver den uændret. 
  \end{lemma}

  \begin{proof}
    Vi beviser invarianten $v.d \geq \delta(s,v)$ med induktion over antallet af relaxations.\\
    Base-case er sandt, siden $v.d = \infty \Rightarrow v.d \geq \delta(s,v)$.\\
    For det induktive trin kalder vi relax på en kant $(u,v)$, hvor vi antager at $v.d > \delta(s, u) + w(u,v)$, da relax ellers ikke ændrer noget og er derfor sandt.\\
    \begin{align*}
      v.d &= u.d + w(u,v)&\\
          &\geq \delta(s,u) + w(u,v) &\text{ (Det induktive trin)}\\
          &\geq \delta(s,v)          &\text{ (Trekants uligheden)}
    \end{align*}

    Grunden til $v.d$ ikke ændrer sig når den når sin nedre grænse er, at når $v.d = \delta(s,v)$, kan den ikke blive lavere, da $\delta(s,v)$ er den korteste vægtede vej. Den kan heller ikke stige, da relax kun ændrer værdien hvis $v.d > \delta(s,u) + w(u,v)$.
  \end{proof}


\item \textbf{Ingen vej egenskaben}
  \begin{corollary}
    Lad $G = (V,E))$ være en orienteret graf med vægtsfunktion $w(u,v)$ på en kant $(u,v) \in G.E$. Lad $s$ være startknuden. Lad $s$ være startknuden. Hvis der ikke findes en vej fra $s$ til en anden knude $v \in G.V$, så har vi at $v.d$ er initialiseret til $v.d = \infty = \delta(s,v)$, og den ændrer sig ikke over alle relax trin på kanterne i $G.E$.
  \end{corollary}

  \begin{proof}
    På grund af den øvre grænse egenskab, har vi at $\infty = \delta(s,v) \leq v.d$ og derfor må $v.d = \infty$, og kan ikke blive lavere.
  \end{proof}

  \begin{lemma}
    Lad $G = (V,E))$ være en orienteret graf med vægtsfunktion $w(u,v)$ på en kant $(u,v) \in G.E$. Lad $s$ være startknuden. Lad $(u,v) \in G.E$. Lige efter et kald på relax $(u,v)$ gælder at $v.d \leq u.d + w(u,v)$.
  \end{lemma}

  \begin{proof}
    Hvis kaldet til relax ændrer $v.d$, så må det gælde før kaldet, at $v.d > u.d + w(u,v)$, derfor må $v.d = u.d + w(u,v)$ efter kaldet. Hvis vi ikke ændrer $v.d$ efter kaldet, må det i forvejen gælde, at
    $v.d \leq u.d + w(u,v)$.
  \end{proof}

\item \textbf{Konvergens egenskaben}
  \begin{lemma}
    Lad $G = (V,E))$ være en orienteret graf med vægtsfunktion $w(u,v)$ på en kant $(u,v) \in G.E$. Lad $s$ være startknuden og lad
    $$s \leadsto u \rightarrow v$$
    være den korteste vej i $G$ for knuderne $u$ og $v$. Knuderne blivere initialiseret, og hvis efter en række relax trin, at $u.d = \delta(s,u)$, vil $v.d = \delta(s,v)$ efter det næste relax trin, og alle trin efter.
  \end{lemma}

  \begin{proof}
    Vi ved fra den øvre grænse egenskab, at hvis $u.d = \delta(s,v)$, vil den forblive sådan efter alle fremtidige relax kald. Efter et relax kald på kanten $(u,v)$ har vi
    \begin{align*}
      v.d &\leq u.d + w(u,v) &\text{ (Fra Lemma 7.4)}\\
          &= \delta(s,u) + w(u,v)&\\
          &= \delta(s,v)     &\text{ (Fra Lemma 7.1, } p_{uv} \subset p_{sv})
    \end{align*}
    Hvis $v.d \leq \delta(s,v)$, må $v.d=\delta(s,v)$.
  \end{proof}


\item \textbf{Vej-relaxation egenskaben}
  \begin{lemma}
    Lad $G = (V,E))$ være en orienteret graf med vægtsfunktion $w(u,v)$ på en kant $(u,v) \in G.E$. Lad $s$ være startknuden. Hvis $p = \langle v_0, v_1, ..., v_k\rangle$, hvor $v_0 = s$, er den korteste vej fra $s \leadsto v_k$. Hvis vi, efter at initialiseret, har relaxet kanterne $(v_0, v_1), (v_1, v_2), ..., (v_{k-1}, v_k)$, så vil $v_k.d = \delta(s,v_k)$ efter det sidste relax kald, og alle fremtidige kald. Denne egenskab gælder også hvis relax bliver kaldt på andre kanter i $e \in G.E$.
  \end{lemma}

  \begin{proof}
    Vi viser ved induktion, at efter $i$ relax kald på $p$, så er $v_i.d = \delta(s,v_i)$.\\
    Base case er når $i=0$, så har vi at $v_0.d = s.d = 0 = \delta(s,s)$.
    Nu antager vi at $v_{i-1}.d = \delta(s, v_{v - 1})$, og vi kalder relax på kanten $(v_{i-1}, v_i)$. På grund af konvergens egenskaben, vil $v_i.d = \delta(s,v_i)$ efter relax kaldet.
  \end{proof}


\item\textbf{Forgænger delgraf egenskaben}
  \begin{lemma}
  \end{lemma}

  \begin{proof}
  \end{proof}

\end{itemize}

\subsection{Bellman-Ford}
Bellman-Ford algoritmen løser korteste vej problemet hvor vi tillader negative vægte. Algoritmen initialiserer alle knuderne, og derefter kører den et loop $|G.V| - 1$ og for hvert loop kalder den relax på alle kanter $(u,v) \in G.E$. Altså bliver hver kant relaxet $|V| - 1$ gange, og til sidst tjekker algoritmen alle kanter for
$$v.d > u.d + w(u,v)$$

Hvis en kant opfylder dette, så må det betyde, at grafen indeholder en negativ cykel, og algoritmen returnerer \texttt{FALSE}, ellers returnerer den \texttt{TRUE}, og den korteste vej kan konstrueres ved at følge en kants forælder -- $v.\pi$.\\

\subsubsection{Køretid}
af Bellman-Ford er $Ø(VE)$ da initialisering tager $\Theta(E)$ og vi går igennem alle $\Theta(E)$ kanter $|V| - 1$ gange, altså $O(VE)$. Det sidste loop er $Ø(E)$.\\

Korrektheden viser vi, ved at vise at tjekket vi laver i det sidste loop er korrekt.
\begin{lemma}
  Lad $G = (V,E)$ være en orienteret graf med vægtsfunktion $w(u,v)$ på kanter $(u,v) \in G.E$. Lad $s$ være startknuden.\\
  Vi antager at $G$ ikke indeholder negative cykler, som kan nås fra $s$. Så efter $|V| - 1$ relaxations af alle kanter, har vi at $v.d = \delta(s,v)$ for alle knuder $v$ der kan nås fra $s$.
\end{lemma}

\begin{proof}
  Vi bruger vej-relaxations egenskaben for grafen $G$. lad $p \langle s, v_1, v_2 , ..., v_k\rangle$ være den korteste vej $s \leadsto v_k$. Fordi korteste veje er simple, er der højst $|V| - 1$ knuder i $p$, og derfor har vi $k \leq |V| - 1$. Efter $i$ relaxationskald, har vi, på grund af vej-relaxationsegenskaben, at $v_k.d = \delta(s,v_k)$.
\end{proof}

\subsubsection{Korrekthed}
\begin{theorem}[Korrekthed af Bellman-Ford]
  
  Vi kører Bellman-Ford på en orienteret graf $G = (V, E)$ med vægtsfunktion $w : G.E \rightarrow \mathbb{R}$. Lad $s$ være startknuden.\\
  Hvis $G$ ikke har nogen negative cykler, så returnerer Bellman-Ford \texttt{TRUE}, og $v.d = \delta(s,v)$ for alle $v \in G.V$, og forgænger delgrafen $G_\pi$udgør et træ med rod $s$.\\
  Hvis $G$ har negative cykler, der kan nås fra $s$, returnerer Bellman-Ford \texttt{FALSE}.
\end{theorem}

\begin{proof}
  Vi kigger på første tilfælde, hvor $G$ ikke har negative cykler.\\
  Vi har at $v.d = \delta(s,v)$. Hvis $s \leadsto v$, så beviser lemma 7.8 dette, ellers hvis der ikke er en vej fra $s$ til $v$, så beviser ingen vej egenskaben, at det også gælder. Forgænger delgraf egenskaben viser at $G_\pi$ er korteste vej træet.\\
  Nu viser vi at Bellman-Ford returnerer \texttt{TRUE}, ved at vise
  \begin{align*}
    v.d &= \delta(s,v)\\
        &\leq \delta{s,u} + w(u,v) \text{ (Ved trekants uligheden)}\\
        &= u.d + w(u,v)
  \end{align*}

  Antage at der findes en negativ cykel, så vil vi vise at Bellman-Ford returnerer \texttt{FALSE}.\\
  Lad cyklen være $c = \langle v_0, v_1, ..., v_k\rangle$ hvor $v_0 = v_k$, så har vi
  $$\sum_{i = 1}^k w(v_{i-1},v_i) < 0$$

  Vi antager nu at Bellman-Ford returnerer \texttt{TRUE}, så gælder $v_i.d \leq v_{i-1}.d + w(v_{i-1},v_i)$, for alle $i = 1, 2, ..., k$. Hvis vi summere alle ulighederne, har vi
  \begin{align*}
    \sum_{i=1}^k v_i.d &\leq \sum_{i=1}^k (v_{i-1}.d + w(v_{i-1}, v_i))\\
                       &= \sum_{i=1}^k v_{i-1}.d + \sum_{i=1}^k w(v_{i-1}, v_i)\\
  \end{align*}

  Da $v_0 = v_k$ har vi også at
  $$\sum_{i=1}^k v_i.d = \sum_{i=1}^k v_{i-1}.d \Rightarrow \sum_{i=1}^k v_i.d - \sum_{i=1}^k v_{i-1}.d = 0$$

  altså
  $$0 \leq \sum_{i=1}^k w(v_{i-1}, v_i)$$
  Hvilket er i modstrid med at cyklen har negativ vægt. Altså, en korteste vej med negativ vægt kan ikke returnere \texttt{TRUE}, og må derfor returnerer \texttt{FALSE}.
\end{proof}

\subsection{Dijkstra}
Dijkstras algoritme finder den korteste vej på grafer \underline{uden} negative vægte, altså $w(u,v) \geq 0$ for alle $(u,v) \in G.E$.\\

Algoritmen holder en mængde $S$ af knuder, hvis korteste vej er fundet, og for iteration vælger den en ny knude i $v \in V-S$ med det laveste estimat $v.d.$. I implementationen bruger vi en min-prioritets kø til at kunne tage en knude med minimum $v.d$ værdi.\\

Først initialiseres alle knuder, $S = \emptyset$, og $Q=G.V$. Herefter trækker vi en knude $u$ fra $Q$ med minimums estimat. Som starter ud med at være $s.d = 0$. Vi tilføjer den til $S$, og så relaxer vi alle kanter $(u,v)$ for $v \in u.adj$, hvor $adj$ er knudens naboliste.\\
Algoritmen holder en invariant $Q = G.V-S$ ved begyndelsen af hver iteration.\\

\subsubsection{Korrekthed}
\begin{theorem}[Korrekthed af Dijkstras algoritme]
  Vi kører Dijkstras algoritme på en orienteret graf $G = (V, E)$ med vægtsfunktion $w : G.E \rightarrow \mathbb{R}_{\geq 0}$. Lad $s$ være startknuden.\\
  Algoritmen slutter med $u.d = \delta(s,u)$ for alle knuder $v \in G.V$.
\end{theorem}

\begin{proof}
  Vi beviser det med følgende invarians:\\
  \begin{center}
    Ved begyndelsen af hver iteration, er $v.d = \delta(s,v)$ for $v \in S$.
  \end{center}

  Vi viser at når vi tilføjer en knude $u \in G.V$ til $S$, så er $u.d = \delta(s,u)$, da vi ved fra den øvre grænse egenskab, at den ikke vil ændre sig efter.\\

  \begin{itemize}
  \item \textbf{Initialisering}: I starten er $S=\emptyset$ og invarianten er trivielt sand.
  \item \textbf{Vedligeholdelse}: Vi ønsker at vise $u.d = \delta(s,u)$ når vi tilføjer $u$ til $S$. Lad os antage at $u$ er den første knude hvor $u.d \neq \delta(s,u)$ når vi tilføjer den til $S$. Vi ved at $u \neq s$ fordi $s$ er allerede i $S$, og fordi $S \neq \emptyset$, må der være en vej fra $s$ til $u$, ellers ville $u.d = \infty = \delta(s,u)$ ifølge ingen vej egenskaben.\\

    Siden der er en vej fra $s$ til $u$, må der også være en korteste vej fra $s$ til $u$. Før $u$ kommer i $S$, der der en knude $x \in S$ og en knude $y \not\in S$, således at $s \leadsto x \rightarrow y \leadsto u$.\\
    Vi påstår at $y.d = \delta(s,y)$ når $u$ kommer i $S$. Vi ved at $x.d = \delta(s,x)$, og på grund af konvergens egenskaben har vi at $(x,y)$ blev relaxed, og derfor er $y.d = \delta(s,y)$. Fordi $y$ kommer før $u$, har vi
    \begin{align*}
      y.d &= \delta(s,y)\\
          &\leq \delta(s,u)\\
          &\leq u.d \text{ (Ved øvre grænse egenskaben)}
    \end{align*}

    Men fordi vi vælger den knude med lavest estimat i $V-S$ gælder også $u.d \leq y.d$, hvilket fører til $y.d = \delta(s,y) = \delta(s,u) = u.d$, hvilket er i modstrid med, at $u.d \neq \delta(s,u)$.\\
    Derfor må $u.d = \delta(s,u)$ når vi tilføjer $u$ til $S$.
    \item \textbf{Afslutning}: Når vi har tilføjet den sidste knude i $S$ er $Q=\emptyset$, hvilket må betyde at $S = V$. Derfor er $u.d = \delta(s,u)$ for alle $u \in G.V$.
  \end{itemize}
\end{proof}

\subsubsection{Køretid}
Vi kan opnå en køretid på $Ø(VlgV + E)$ ved at bruge en Fibonacci hob til at implementere min-prioritetskøen. Insert er theta(lgn) < theta(1), decrease-key er theta(lgn) < theta(1) extract-min theta(lgn) < O(lgn)
%%% Local Variables:
%%% mode: latex
%%% TeX-master: "master"
%%% End:
