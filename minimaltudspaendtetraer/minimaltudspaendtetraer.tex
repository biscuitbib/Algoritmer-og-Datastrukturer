\section{Minimalt Udspændte Træer}
\hrulefill

\begin{itemize}
\item Udspændte træer
\item Kruskals algoritme
  \begin{itemize}
  \item Disjunkte mængder
  \item Køretid af Kruskals algoritme
  \end{itemize}
\item Prims algoritme
  \begin{itemize}
    \item Prioritetskø
  \end{itemize}
\end{itemize}


\newpage
\subsection{Udspændte træer}
givet en graf $G = (V, E)$, så er $T \subset G.E$ et udspændt træ, således at $w(T) = \sum_{e \in T}w(e)$, hvor $w(u,v)$ er en vægt-funktion der bestemmer vægten af en kant der går mellem knuderne $u,v$. Et minimalt udspændt træ er, som navnet hentyder, et udspændt træ for $G$ med den laveste samlede vægt.\\

Vi bruger en grådig algoritme til at bygge et MST, hvor det grådige valg er at tilføje en kant $e \in G.E$ og tilføje den til en mængde $A$. For at tilføje kanter til $A$ holder vi øje med følgende invariant:
\begin{center}
Før vi tilføjer en ny knude, udgør $A$ en delmængde af et MST.
\end{center}

Selve algoritmen går ud på at finde en \textbf{safe edge}, som er en kant $e$ der overholder invarianten og så tilføje den til $A$, og dette gør man iterativt, indtil $A$ udgør et udspændt træ for $G$.\\
\begin{itemize}
\item \textbf{Initiering}: Den tomme mængde er en delmængde af et MST.\\
\item \textbf{Vedligeholdelse}: Vi tilføjer kun safe-edges, og overholder derfor invarianten.\\
\item \textbf{Afslutning}: Siden alle kanterne vi har tilføjet har overholdt invarianten, må alle kanter i $A$ være i et MST, derfor må $A$ være et MST.
\end{itemize}

For at finde en safe-edge skal vi finde en \textbf{light-edge}, som er en minimumsvægtet kant der forbinder et \textbf{cut} af $G$. Et cut er en opdeling af $G.V$, således at $G$ bliver splittet i to nye disjunkte grafer indeholdende knuderne $S$ og $V-S$.
\begin{theorem}
  Lad $G=(V,E)$ være en forbundet graf med en vægtsfunktion $w(u,v)$ i $G.E$. Lad $A$ være en delmængde af $G.E$, som er inkluderet i et MST for $G$.\\
  Lad $(S, V-S)$ være et cut af $G$ der respekterer $A$, og lad $(u,v)$ være en light-edge som forbinder det cut, så vil $(u,v)$ være en safe-edge for $A$.
\end{theorem}


\begin{proof}
  Vi beviser det med modstrid. Lad $T$ være et MST der indeholder $A$, og som ikke indeholder en light-edge $(u,v)$. Vi bygger nu et nyt MST $T'$, som indeholder $A \cup {(u,v)}$, og derved vise at $(u,v)$ er en safe-edge.\\

  $(u, v)$ forbinder cuttet $(S,V-S)$, og siden $G$ er forbundet, må der være en kant $(x,y)$ fra $u \leadsto v$ der også forbinder cuttet $(S,V-S)$. $(x,y)$ er ikke i $A$. Vi laver nu et nyt træ, vi bruger $(u,v)$ til at forbinde cuttet i stedet for $(x,y)$:
  $$T' = T - \{(x,y)\} \cup \{(u,v)\}$$
  
  Siden $(u,v)$ er en light-edge for cuttet, så må $w(u,v) \leq w(x,y)$, hvilket betyder
  \begin{align*}
    w(T') &= w(T) - w(x,y) + w(u,v)\\
          &\leq w(T)
  \end{align*}

  Derfor må $T'$ også være et MST for $G$. Det sidste vi mangler at vise er, at $(u,v)$ er en safe-edge for $A$, hvilket kan ses ved at $A \cup \{(u,v)\} \subset T'$.
\end{proof}

\subsection{Kruskals algoritme}
Den første algoritme til at finde MST er Kruskals algoritme. Denne algoritme bruger en \textbf{skov} af træer, som er alle knuderne $G.V$. Derefter forbinder den de knuder som udgør en safe-edge.

\subsubsection{Disjunkte mængder}
Kruskals algoritme bruger disjunkte mængder til at holde kanterne i MST.\\

Disjunkte mængder, er en datastruktur der understøtter følgende operationer og egenskaber.\\

\begin{itemize}
\item Hver mængde har et kanonisk element, kaldet mængdens repræsentant.
\item \texttt{Make-Set($x$)} laver en ny mængde med $x$ som repræsentant.
\item \texttt{Find-Set($x$)} returnerer repræsentanten for mængden der indeholder elementet $x$.
\item \texttt{Union($x,y$)} Forbinder de to mængder med repræsentanter $x$ og $y$, således at den nye mængdes repræsentant er en af de to.
\end{itemize}

Der er to måder vi kan implementere disjunkte mængder på: Enten kan vi bruge en \textbf{linked-list}repræsentation, hvor hver mængde er en linked-list, med et mængde objekt, der peger på repræsentanten og hvor alle elementer peger på. Den anden metode, som vi bruger til Kruskal, er \textbf{Disjunkt-mængde skove}, som er en skov af disjunkte træer, hvor roden er repræsentanten af mængden.\\

De tre operationer kan implementeres på følgende måde:
\begin{itemize}
\item \texttt{Make-Set($x$)}: Vi laver et nyt træ, kun indeholdende $x$, og vi sætter $x.\pi$ til at være sig selv, derved er den roden. Dette tager $Ø(1)$ tid.
\item \texttt{Find-Set($x$)}: Vi følger $x = x.\pi$ op indtil $x = x.\pi$, hvorefter vi returnerer repræsentanten. Dette tager $Ø(h)$ tid.
\item \texttt{Union($x,y$)}: Vi indsætter roden af det ene træ ind i det andet træ. Dette kan i værste fald føre til en hægtet-liste af elementer, derfor indføre vi to heuristikker: \textbf{Union by rank} og \textbf{path compression}. I union by rank holder hver knude et estimat af højden af dets deltræ, dets \textbf{rank}, og derfor kan vi forbinde det mindre træ på det større træ. Path compression indføres i \texttt{Find-Set($x$)}, hvor, så længe $x \neq x.\pi$ sætter vi $x.\pi = \texttt{Find-Set($x$)}$, dette minimerer højden af træet, og gør forældrestigen så kort som muligt.
\end{itemize}

\subsubsection{Køretid af Kruskals algoritme}
Vi bruger \textbf{disjoint-set forets} til at udvikle MST med Kruskals.\\
Første trin er at oprette en skov af mængder indeholdene alle knuderne i $G.V$, derefter sorteres alle kanterne efter stigende vægt. Derefter looper vi over alle kanterne $(u,v)$, og så længe $\texttt{Find-Set($u$)} \neq \texttt{Find-Set($v$)}$ så forbinder vi mængderne med \texttt{Union($u,v$)}. Til sidst returnerer vi den disjunkte mængde, som, hvis $G$ er forbundet, er et MST.\\

Køretiden af Kruskals algoritme er $Ø(ElgV)$. Først laver vi $|V|$ disjunkte træer, som er $Ø(V)$, herefter looper vi over alle kanterne, og forbinder kantens knuder. Siden vi 


\subsection{Prims algoritme}
Prims algoritme fungerer meget som Dijkstras korteste vej algoritme. Her har vi en prioritetskø af $v.key$, som er en værdi der er den mindste vægt af en kant der forbinder til $v$. Vi initialiserer alle $v.key = \infty$ og $v.\pi$, derefter vælges en arbitrær knude $r$ til roden, og $r.key = 0$. Derefter lægges alle knuder ind i prioritetskøen $Q = G.V$. Vi udvælger knuden $u$ med den mindste $key$ i $Q$, som starter med at være $r$ og tjekker alle dens naboer $v$, og retter $v.key = w(u,v)$ hvis $w(u,v) < v.key$, og sætter $v.\pi = u$. Når køen er tom, betyder det at alle knuderne er i træet, og vi kan eventuelt returnerer rod-knuden.\\

Algoritmen overholder følgende invarianter:
\begin{enumerate}
\item $A = \{(v,v.\pi) : v \in G.V - \{r\} - Q\}$
\item Alle knuder i MST er knuder i $G.V - Q$
\item For alle knuder $v \in G.V$, hvis $v.key < \infty$, så er $v.\pi \neq \texttt{NIL}$ og $(v, v.\pi)$ er en light-edge der forbinder $v$ til en knude der allerede er i MST.
\end{enumerate}

Køretiden af Prims algoritme er $Ø(E+VlgV)$ hvis vi implementerer vores prioritetskø med en Fibonacci hob, da \texttt{EXTRACT-MIN} tager $Ø(lgV)$, som udføres hvert loop $Ø(V)$ gange, og \texttt{DECREASE-KEY} operation tager $Ø(1)$ tid og udføres $Ø(E)$ gange. Dette giver $Ø(E+VlgV)$. 

%%% Local Variables:
%%% mode: latex
%%% TeX-master: "master"
%%% End:
